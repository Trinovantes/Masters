\chapter{Conclusion}
\label{chap:conclusion}

This thesis presented a novel approach to automatically mark programming assignments. Our approach consists of two components: (1) we first prune a program's AST to isolate key features relevant for assignment marking; (2) we then compare a student solution to a set of reference solutions in order to generate a final mark for the student. Assignments that have received automated deductions will require manual review to provide more meaningful and individualized feedback.

This tool assumes the majority of the students will write their programs in specific patterns. We are concerned this may limit student creativity when trying to solve their assignments. However, we note that the programming assignments we tested with had rigid designs and the easiest-to-implement solutions often fall under specific patterns. Therefore we do not believe this to be a major concern.

We implemented our processes as the ClangAutoMarker tool and tested it with student submissions and marks from previous offerings of the ECE459 course at the University of Waterloo. Our initial results were not as successful as we had originally hoped. Our tool did not perform better than the baseline approach of simply always assigning full marks. However, due to the uncertainty in our ground-truth, we faithfully recollected the ground-truth data for a smaller subset of previous classes. When we reevaluated our tool with the more accurate sample, we were able to achieve a better false positive rate of 21\% compared to always assigning full marks which had a false positive rate of 35\%. Although this was still not very accurate, we have demonstrated that our tool has promising potential for automated marking and further improvements may make it viable for a live classroom.
